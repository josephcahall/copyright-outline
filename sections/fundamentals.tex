\section{Fundamentals of the U.S. Copyright System}

\subsection{History of U.S. Copyright Law}

% TODO 21-42
% TODO supp 357-61

\subsection{Subject Matter}

\begin{enumerate}
    \item 17 U.S.C. \S\ 102(a): ``Copyright protection subsists, in accordance 
    with this title, in original works of authorship fixed in any tangible 
    medium of expression, now known or later developed, from which they can be 
    perceived, reproduced, or otherwise communicated, either directly or with 
    the aid of a machine or device. Works of authorship include the following 
    categories~.~.~.~.''
    \item I.e., there are three statutory subject matter requirements:
    \begin{enumerate}
        \item ``\textbf{fixed} in any tangible medium of expression''
        \item ``\textbf{original} works of authorship''
        \item No protection for \textbf{ideas, procedures, etc.} (see \S\ 
        102(b)).
    \end{enumerate}
\end{enumerate}

\subsubsection{Fixation}

\begin{enumerate}
    \item Not all countries require fixation (compared to originality, which 
    is universally required). So why require it?
    \begin{enumerate}
        \item It's useful as evidence in a dispute.
        \item The Constitution refers to ``writings'' (although some sections 
        of the act---e.g., \S\ 1101, referring to live performances---are 
        authorized under the Commerce Clause).
    \end{enumerate}
    \item Common-law copyright \emph{may} apply to unfixed works.
    \item Fixation helps determine:
    \begin{enumerate}
        \item Is the work \textbf{protectable}?
        \item Has the \textbf{reproduction right} been infringed?
    \end{enumerate}
\end{enumerate}

\paragraph{Functional Approach}

\begin{enumerate}
    \item \S\ 101: ``A work is `fixed' in a tangible medium of expression when 
    its embodiment in a copy or phonorecord, by or under the authority of the 
    author, is sufficiently permanent or stable to permit it to be perceived, 
    reproduced, or otherwise communicated for a period of more than transitory 
    duration. A work consisting of sounds, images, or both, that are being 
    transmitted, is `fixed' for purposes of this title if a fixation of the 
    work is being made simultaneously with its transmission.''
    \item I.e., the statute is \textbf{technology-neutral} (but note the 
    distinction between copies and phonorecords).
\end{enumerate}

\paragraph{Ephemeral Images in a Video Game: \emph{Williams Electronics, Inc. 
v. Artic International, Inc.}}

\begin{enumerate}
    \item The developer of a knock-off arcade game argued that the original 
    game lacked protection because the images and sounds it displayed weren't 
    ``fixed.''\footnote{Casebook p. 48--50.}
    \item Held: even though the audiovisual displays varied depending on the 
    user's interaction, ``there is always a repetitive sequence of a 
    substantial portion of the sights and sounds of the game, and many aspects 
    of the display remain constant~.~.~.~.''\footnote{Casebook p. 50.}
\end{enumerate}

\paragraph{Software in RAM: \emph{MAI Systems Corp. v. Peak Computer, Inc.}}

\begin{enumerate}
    \item MAI, a software maker, sought to prevent Peak, a third-party repair 
    service provider, from selling support services for its operating. MAI's 
    theory was that Peak created unauthorized copies of the software by 
    loading it into the RAM of its customers' machines.\footnote{Casebook p. 
    50--55.}
    \item Held: copies of a program in RAM are sufficiently fixed to qualify 
    for protection.
    \item (Later overruled in the DMCA.)
\end{enumerate}

\paragraph{Technology-Specific Approach}

\begin{enumerate}
    \item Second sentence of the definition of ``fixed''in \S\ 101: ``A work 
    consisting of sounds, images, or both, that are being transmitted, is 
    `fixed' for purposes of this title if a fixation of the work is being made 
    simultaneously with its transmission.''
    \item Loophole: live performances aren't ``transmissions,'' so they could 
    be bootlegged. Loophole fixed in 1994 in \S\ 1101(a) (under Commerce 
    Clause authority, since a performance isn't a 
    ``writing'').\footnote{Casebook p. 56--57.}
    \item Another loophole? \S\ 1101(a) only applies to ``musical 
    performances''---so what about, e.g., stand-up comedy?
\end{enumerate}

\subsubsection{Originality}

\begin{enumerate}
    \item Three functions of originality:
    \begin{enumerate}
        \item Is a work \textbf{eligible for protection}?
        \item What is the \textbf{scope of protection}?
        \item If someone copies another's work, but copies an unoriginal 
        component, infringement may not have occurred---e.g., if Microsoft 
        copies a GUI element from Xerox, then Apple can't be an infringer if 
        it copies those elements from Microsoft's GUI.
    \end{enumerate}
\end{enumerate}

\paragraph{The Modern U.S. Definition of Originality: \emph{Feist 
Publications, Inc. v. Rural Telephone Service Co.}}

\begin{enumerate}
    \item ``Original, as the term is used in copyright, means only that the 
    work was independently created by the author (as opposed to copied from 
    other works), and that it possesses at least some minimal degree of 
    creativity.''\footnote{Casebook p. 59.}
    \item Facts are generally not protected, but compilations can be if they 
    are ``selected, coordinated, or arranged in such a way that the resulting 
    work as a whole constitutes an original work of 
    authorship.''\footnote{Casebook p. 60.}
    \item The reason, according to the court, is that \textbf{facts are 
    discovered}, not created. (But is this really the case? Maybe a better 
    standard would be: copyright doesn't apply to facts because people need to 
    be able to reuse facts in order to engage in the promotion of science.)
    \item (The cases below adopted alternative theories of originality. 
    \emph{Feist} replaced them.)
\end{enumerate}

\paragraph{\emph{Burrow-Giles Lithographic Co. v. Saron}}

\begin{enumerate}
    \item Are photographs protectable? Yes.
    \item Originality derives from the author's mind. The selection and 
    arrangement of the elements in the photo constitute original 
    expression.\footnote{Casebook p. 61--63.}
\end{enumerate}

\paragraph{\emph{Bleistein v. Donaldson Lithographing Co.}}

\begin{enumerate}
    \item Are advertisements protectable? Yes.\footnote{Casebook p. 64--67.}
    \item The \textbf{aesthetic nondiscrimination principle}: judges are not 
    qualified to assess aesthetic merit.
    \item The standard for originality is the author's ``personal imprint.''
\end{enumerate}

\paragraph{\emph{Alfred Bell \& Co. v. Catalda Fine Arts, Inc.}}

\begin{enumerate}
    \item Mezzotint engravings, which reproduced public domain paintings, were 
    protectable because no two engravings were alike. Artistic decisions were 
    required.\footnote{Casebook p. 67.}
    \item This case is mostly still good law---apart from the language about 
    variations due to thunderclaps, etc., being protectable (because, after 
    \emph{Feist}, protectability requires a modicum of creativity).
\end{enumerate}

\paragraph{Nonobviousness and Originality}

% TODO 70-72

\paragraph{\emph{Meshwerks, Inc. v. Toyota Motor Sales U.S.A., Inc.}}

% TODO 72-76
\begin{enumerate}
    \item Authorial intent matters.
\end{enumerate}

\subsection{Idea/Expression Distinction}

% TODO 81-94
